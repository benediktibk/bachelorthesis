\chapter{Introduction}

Currently a lot of new chips are designed and manufactured every single day. A lot of them are mainly digital, as our modern world becomes more and more digital as well. But some parts of these digital chips are still analog, even if it is not desired. The problem is that the surroundings of a chip will always be analog, so at least some circuit elements must also be viewed as analog. The analog world still lacks the tools to fully automatize the steps from an idea to the actual layout of a chip for manufacturing. Some tools for placing and some implementations for routing algorithms are available. During the last few years a completely new approach for the placement was developed, the hierarchical placer called Plantage \cite{iccad:plantage}, and now it is time to implement the next step towards a fully automatized analog layout tool into this software: routing.

The actually last missing step will be the integration of the already developed automatic circuit analysis. This tool could automatically generate the constraints from the schematic and therefore increases the degree of automation. The user will certainly still have to modify these suggested constraints but the main work should already be done by the tool.

As we already have a fast and reliable placer, I have implemented the routing into this tool. The main results of this thesis are:
\begin{itemize}
\item a line expansion router, which creates, in most cases, design rule clean routes and considers symmetries
\item based on the previous router an extended version, which tries to reduce the parasitics
\item a software design, where the routing algorithm can be exchanged easily
\end{itemize}